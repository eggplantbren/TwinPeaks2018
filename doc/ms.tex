%  LaTeX support: latex@mdpi.com 
%  In case you need support, please attach all files that are necessary for compiling as well as the log file, and specify the details of your LaTeX setup (which operating system and LaTeX version / tools you are using).

% You need to save the "mdpi.cls" and "mdpi.bst" files into the same folder as this template file.

%=================================================================
\documentclass[entropy,article,submit,moreauthors,pdftex,10pt,a4paper]{Definitions/mdpi} 

\usepackage{dsfont}

% If you would like to post an early version of this manuscript as a preprint, you may use preprint as the journal and change 'submit' to 'accept'. The document class line would be, e.g. \documentclass[preprints,article,accept,moreauthors,pdftex,10pt,a4paper]{mdpi}. This is especially recommended for submission to arXiv, where line numbers should be removed before posting. For preprints.org, the editorial staff will make this change immediately prior to posting.

%
%--------------------
% Class Options:
%--------------------
% journal
%----------
% Choose between the following MDPI journals:
% acoustics, actuators, addictions, admsci, aerospace, agriculture, agronomy, algorithms, animals, antibiotics, antibodies, antioxidants, applsci, arts, asi, atmosphere, atoms, axioms, batteries, bdcc, behavsci, beverages, bioengineering, biology, biomedicines, biomimetics, biomolecules, biosensors, brainsci, buildings, carbon, cancers, catalysts, cells, ceramics, challenges, chemengineering, chemosensors, children, cleantechnol, climate, clockssleep, cmd, coatings, colloids, computation, computers, condensedmatter, cosmetics, cryptography, crystals, cybersecurity, data, dentistry, designs, diagnostics, dairy, diseases, diversity, drones, econometrics, economies, education, electrochem, electrochemistry, electronics, energies, entropy, environments, epigenomes, est, fermentation, fibers, fire, fishes, fluids, foods, forecasting, forests, fractalfract, futureinternet, galaxies, games, gastrointestdisord, gels, genealogy, genes, geohazards, geosciences, geriatrics, hazardousmatters, healthcare, heritage, highthroughput, horticulturae, humanities, hydrology, informatics, information, infrastructures, inorganics, insects, instruments, ijerph, ijfs, ijms, ijgi, ijtpp, inventions, j, jcdd, jcm, jcs, jdb, jfb, jfmk, jimaging, jof, jintelligence, jlpea, jmmp, jmse, jpm, jrfm, jsan, land, languages, laws, life, literature, logistics, lubricants, machines, magnetochemistry, make, marinedrugs, materials, mathematics, mca, medsci, medicina, medicines, membranes, metabolites, metals, microarrays, micromachines, microorganisms, minerals, modelling, molbank, molecules, mps, mti, nanomaterials, ncrna, neonatalscreening, neuroglia, nitrogen, nutrients, ohbm, particles, pathogens, pharmaceuticals, pharmaceutics, pharmacy, philosophies, photonics, plants, plasma, polymers, polysaccharides, proceedings, processes, proteomes, publications, quaternary, qubs, reactions, recycling, religions, remotesensing, reports, resources, risks, robotics, safety, sci, scipharm, sensors, separations, sexes, sinusitis, smartcities, socsci, societies, soilsystems, sports, standards, stats, surfaces, surgeries, sustainability, symmetry, systems, technologies, toxics, toxins, tropicalmed, universe, urbansci, vaccines, vehicles, vetsci, vibration, viruses, vision, water, wem, wevj
%---------
% article
%---------
% The default type of manuscript is article, but can be replaced by: 
% abstract, addendum, article, benchmark, book, bookreview, briefreport, casereport, changes, comment, commentary, communication, conceptpaper, correction, conferenceproceedings, conferencereport, expressionofconcern, meetingreport, creative, datadescriptor, discussion, editorial, essay, erratum, hypothesis, interestingimages, letter, meetingreport, newbookreceived, opinion, obituary, projectreport, reply, reprint, retraction, review, perspective, protocol, shortnote, supfile, technicalnote, viewpoint
% supfile = supplementary materials
% protocol: If you are preparing a "Protocol" paper, please refer to http://www.mdpi.com/journal/mps/instructions for details on its expected structure and content.
%----------
% submit
%----------
% The class option "submit" will be changed to "accept" by the Editorial Office when the paper is accepted. This will only make changes to the frontpage (e.g. the logo of the journal will get visible), the headings, and the copyright information. Also, line numbering will be removed. Journal info and pagination for accepted papers will also be assigned by the Editorial Office.
%------------------
% moreauthors
%------------------
% If there is only one author the class option oneauthor should be used. Otherwise use the class option moreauthors.
%---------
% pdftex
%---------
% The option pdftex is for use with pdfLaTeX. If eps figures are used, remove the option pdftex and use LaTeX and dvi2pdf.

%=================================================================
\firstpage{1} 
\makeatletter 
\setcounter{page}{\@firstpage} 
\makeatother
\pubvolume{xx}
\issuenum{1}
\articlenumber{1}
\pubyear{2018}
\copyrightyear{2018}
\externaleditor{Academic Editor: name}
\history{Received: date; Accepted: date; Published: date}

%\updates{yes} % If there is an update available, un-comment this line
 
%------------------------------------------------------------------
% The following line should be uncommented if the LaTeX file is uploaded to arXiv.org
%\pdfoutput=1

%=================================================================
% Add packages and commands here. The following packages are loaded in our class file: fontenc, calc, indentfirst, fancyhdr, graphicx, lastpage, ifthen, lineno, float, amsmath, setspace, enumitem, mathpazo, booktabs, titlesec, etoolbox, amsthm, hyphenat, natbib, hyperref, footmisc, geometry, caption, url, mdframed, tabto, soul, multirow, microtype, tikz

%=================================================================
%% Please use the following mathematics environments: Theorem, Lemma, Corollary, Proposition, Characterization, Property, Problem, Example, ExamplesandDefinitions, Hypothesis, Remark, Definition
%% For proofs, please use the proof environment (the amsthm package is loaded by the MDPI class).

%=================================================================
% Full title of the paper (Capitalized)
\Title{Nested Sampling with Multiple Objective Functions}

% Author Orchid ID: enter ID or remove command
\newcommand{\xx}{\boldsymbol{\theta}}
\newcommand{\dx}{d\boldsymbol{\theta}}
\newcommand{\data}{D}
\newcommand{\II}{I}
\newcommand{\mnras}{Monthly Notices of the Royal Astronomical Society}

%\newcommand{\orcidauthorA}{0000-0000-000-000X} % Add \orcidA{} behind the author's name
%\newcommand{\orcidauthorB}{0000-0000-000-000X} % Add \orcidB{} behind the author's name

% Authors, for the paper (add full first names)
\Author{Brendon J. Brewer$^{1,}$*,
        Robert J. N. Baldock$^2$ and
        Ewan Cameron$^3$}

% Authors, for metadata in PDF
\AuthorNames{Brendon J. Brewer, Robert J. N. Baldock and Ewan Cameron}

% Affiliations / Addresses (Add [1] after \address if there is only one affiliation.)
\address{%
$^{1}$ \quad Department of Statistics, The University of Auckland, Private Bag 92019,
Auckland 1142, New Zealand \\
$^{2}$ \quad TBD \\
$^{3}$ \quad TBD }

% Contact information of the corresponding author
\corres{Correspondence: bj.brewer@auckland.ac.nz}

% Current address and/or shared authorship
%\firstnote{Current address: Affiliation 3} 
%\secondnote{These authors contributed equally to this work.}
% The commands \thirdnote{} till \eighthnote{} are available for further notes

% Simple summary
%\simplesumm{}

% Abstract (Do not insert blank lines, i.e. \\) 
\abstract{We present a generalization of Skilling's Nested Sampling algorithm
that estimates
the partition function $Z(\beta_1, \beta_2)$ for a family of
probability distributions indexed by two hyperparameters, $\beta_1$ and
$\beta_2$. Problems like this arise frequently in statistical mechanics and
occasionally in Bayesian inference.
The algorithm is based on an interpretation of Nested Sampling in terms of
Sequential Monte Carlo, which produces unbiased estimates of $Z$ and therefore
allows parallel runs to be trivially combined.
We demonstrate the algorithm on three example problems:
(i)
a toy problem whose structure allows us
to accurately calculate $Z(\beta_1, \beta_2)$ numerically using non-Monte-Carlo
methods, so we can compute the actual error in the reconstructed function
$Z(\beta_1, \beta_2)$; (ii) a collection of atoms with a potential
that is intermediate between a Lennard-Jones form and a `polymer' form; and
(iii) A Bayesian image reconstruction example where the prior for the image
is defined conditionally on a hyperparameter, but where this prior contains
an intractable normalizing constant.
A free software implementation in C++ and Python is available at
\url{https://github.com/eggplantbren/TwinPeaks2018}.}

% Keywords
\keyword{nested sampling; bayesian computation; statistical mechanics;
         partition functions;
         sequential monte carlo; multi-objective optimization}

% The fields PACS, MSC, and JEL may be left empty or commented out if not applicable
%\PACS{J0101}
%\MSC{}
%\JEL{}

%%%%%%%%%%%%%%%%%%%%%%%%%%%%%%%%%%%%%%%%%%
% Only for the journal Applied Sciences:
%\featuredapplication{Authors are encouraged to provide a concise description of the specific application or a potential application of the work. This section is not mandatory.}
%%%%%%%%%%%%%%%%%%%%%%%%%%%%%%%%%%%%%%%%%%

%%%%%%%%%%%%%%%%%%%%%%%%%%%%%%%%%%%%%%%%%%
% Only for the journal Data:
%\dataset{DOI number or link to the deposited data set in cases where the data set is published or set to be published separately. If the data set is submitted and will be published as a supplement to this paper in the journal Data, this field will be filled by the editors of the journal. In this case, please make sure to submit the data set as a supplement when entering your manuscript into our manuscript editorial system.}

%\datasetlicense{license under which the data set is made available (CC0, CC-BY, CC-BY-SA, CC-BY-NC, etc.)}

%%%%%%%%%%%%%%%%%%%%%%%%%%%%%%%%%%%%%%%%%%
% Only for the journal Toxins
%\keycontribution{The breakthroughs or highlights of the manuscript. Authors can write one or two sentences to describe the most important part of the paper.}

%\setcounter{secnumdepth}{4}
%%%%%%%%%%%%%%%%%%%%%%%%%%%%%%%%%%%%%%%%%%
\begin{document}
%%%%%%%%%%%%%%%%%%%%%%%%%%%%%%%%%%%%%%%%%%
%% Only for the journal Gels: Please place the Experimental Section after the Conclusions

%%%%%%%%%%%%%%%%%%%%%%%%%%%%%%%%%%%%%%%%%%
%\setcounter{section}{-1} %% Remove this when starting to work on the template.
%\section{How to Use this Template}
%The template details the sections that can be used in a manuscript. Note that the order and names of article sections may differ from the requirements of the journal (e.g. the positioning of the Materials and Methods section). Please check the instructions for authors page of the journal to verify the correct order and names. For any questions, please contact the editorial office of the journal or support@mdpi.com. For LaTeX related questions please contact Janine Daum at latex-support@mdpi.com.
%The order of the section titles is: Introduction, Materials and Methods, Results, Discussion, Conclusions for these journals: aerospace,algorithms,antibodies,antioxidants,atmosphere,axioms,biomedicines,carbon,crystals,designs,diagnostics,environments,fermentation,fluids,forests,fractalfract,informatics,information,inventions,jfmk,jrfm,lubricants,neonatalscreening,neuroglia,particles,pharmaceutics,polymers,processes,technologies,viruses,vision

\setlength{\parskip}{0.6em}

\section{Nested Sampling}

Nested Sampling (NS) \citep{skilling2006nested} is general
Monte Carlo algorithm that can solve a wide range of problems in Bayesian
computation, statistical mechanics, and information theory
\citep[e.g.][]{partay2010efficient, exoplanet, baldock2016determining,
brewer2017computing}.
Its key strength its ability to cope with phase transitions and other
features which cause problems for many other methods such as those based
on `annealing' \citep{neal2001annealed}. These methods all fall into that half
of the Monte Carlo landscape that addresses {\em what distribution to sample}
rather than how to sample it.

In a Bayesian inference problem with unknown parameters denoted collectively
by a vector $\xx$, the
posterior distribution for the parameters given a data proposition $\data$ and
a `prior information' proposition $\II$ is:
\begin{eqnarray}
p(\xx | \data, \II) &=&
\frac{p(\xx | \II)p(\data | \xx, \II)}{p(\data | \II)}\\
&=& \frac{\pi(\xx)L(\xx)}{Z}
\end{eqnarray}
where $\pi(\xx)$ is the prior distribution, $L(\xx)$ is the likelihood
function, and $Z$ is the normalising constant, known as the
`marginal likelihood' or `evidence':
\begin{eqnarray}
Z &=& \int \pi(\xx) L(\xx) \, \dx.\label{eqn:evidence}
\end{eqnarray}

$Z$ is useful because it is the prior probability of the data conditional on
the hypothesis that $\xx$ is {\em one of}
the possibilities in the space being integrated over, along with
whatever additional assumptions were made. $Z$ thus plays the role
of a likelihood for the parameter space as a whole, and is a key input in
calculating the posterior probabilities of competing `models' (mutually
exclusive choices of $\II$).
Naively, Equation~\ref{eqn:evidence} appears straightforward to compute because
it is a simple expectation of $L$ with respect to the distribution $\pi$.
However,
NS is required because Equation~\ref{eqn:evidence}
is the expected value of $L$ with respect to a very heavy-tailed distribution
(the distribution of $L$-values implied by $\pi(\xx)$). This means a very
large number of samples would be required.
Hence, there is a strong
connection between NS and ideas from rare event simulation.
\citet{walter2017point} has explicitly explored this connection.
Another way of thinking about this
challenge is that the $Z$ integral is dominated by a region with very
high values of $L$ but very low probability according to $\pi$ (and usually
very low volume according to $d\xx$, as well).

The main goal of NS is to compute $Z$, but it can also be used to make
Monte Carlo approximations of the posterior distribution, and to
calculate the `information',
or Kullback-Leibler divergence from the prior to the posterior:
\begin{eqnarray}
H &=& \int p(\xx | \data, \II) \log
\left(\frac{p(\xx | \data, \II)}{p(\xx | \II)}\right) \, d\xx \\
&=& \int \frac{\pi(\xx) L(\xx)}{Z} \log
\left(\frac{L(\xx)}{Z}\right) \, d\xx.
\end{eqnarray}
$H$ quantifies how compressed the posterior distribution is with
respect to the prior. For example, $H = 100$ nats (the units obtained when
the logarithm is the natural logarithm) implies, loosely speaking,
that the posterior occupies a fraction $e^{-100}$ of the prior mass.
$H$ can be interpreted quite literally as a measure of how much
the specific data $\data$ resolved the question ``what is the value
of $\xx$, precisely?'' \citep{knuth_questions, van2017inquiry}.
\citet{brewer2017computing} introduced a variant of NS that computes
Shannon entropies, including entropies of marginal distributions (whereas
standard NS can only compute divergences for full joint distributions).

NS works by drawing particles from the
prior $\pi(\xx)$ and successively imposing constraints on the value of
the likelihood $L(\xx)$ that compress the prior mass by a (roughly) known
and constant factor. Like related algorithms such as those
based on `annealing' or `tempering', NS
moves through a sequence of probability distributions, beginning with the
prior. It is the specific sequence of distributions used that distinguishes
NS from annealing. The sequence of distributions used in NS is defined by
\begin{eqnarray}
p(\xx; \ell) &=& \frac{1}{X(\ell)}\pi(\xx)\mathds{1}\left(L(\xx) > \ell\right).
\label{eq:constrained_prior}
\end{eqnarray}
where $X(\ell) = \int \pi(\xx)\mathds{1}\left(L(\xx) > \ell\right)$ is the
normalising constant of the distribution constrained by the condition
$L(\xx) > \ell)$. $X$ is also the quantile function of the likelihoods.

The marginal likelihood can be computed by numerically integrating the
function $L(X)$:
\begin{eqnarray}
Z &=& \int_0^1 L(X)\, dX.
\end{eqnarray}
\citet{salomone2018unbiased}
recently showed how NS can be viewed as an instance of
Sequential Monte Carlo (SMC), with a particular choice of the sequence of
distributions. By weighting the sequence of discarded points differently from
Skilling, the estimates of $Z$ are unbiased.

%Throughout this paper we use the popular `overloaded' notation for some
%functions and probability distributions.
%For example, $L(\xx)$ is the likelihood function, and $L(X)$ is the likelihood
%as a function of the enclosed prior mass $X$. The inverse, $X(L)$, takes a
%likelihood as input and computes the corresponding enclosed prior mass.
%Therefore the symbol $L$ implies the output of the function is a
%likelihood value; so $L(\xx)$ represents a likelihood value computed from
%the parameters $\xx$, and $L(X)$ represents a likelihood value that
%corresponds to an amount $X$ of prior mass. In more traditional notation,
%$X(L)$ would be written using function composition, $X(L(\xx))$.

\section{Accessing distributions other than the posterior}
Using a single NS run, we can also calculate the properties of any
distribution that
is (in some sense which we do not make precise) {\em intermediate}
between the prior and the posterior. For example, we might be interested in
a `power posterior' where the likelihood is raised to a power $\beta$
(this is also the family of distributions used in `annealing' or `tempering'
methods):
\begin{eqnarray}
p(\xx; \beta) &=& \frac{\pi(\xx)L(\xx)^\beta}
                       {Z(\beta)}\label{eqn:power_posterior}
\end{eqnarray}
The normalisation and posterior samples from this distribution can be obtained
from the original NS run by re-weighting the output according to $L(\xx)^\beta$
instead of the usual $L(\xx)$ used to obtain the posterior.
Bayesian inference problems with `gaussian noise' likelihood assumptions
provide an example of where power posteriors might be useful.
Computing $p(\xx; \beta)$ for $\beta \neq 1$ allows us to explore what the
posterior distribution would have been if the noise variance had been greater,
without having to re-run the algorithm.
This is different from including an extra parameter to allow the noise variance
to be greater, because NS allows you to test the consequences of values of the
variance that are very implausible given the data, i.e. that do not have
a high posterior probability.

Alternatives to NS include methods based on `annealing', where a sequence of
distributions of the form of Equation~\ref{eqn:power_posterior} is used.
There are many different methods based on this idea, such as the popular
parallel tempering method \citep{pt, gregory}.
However, unlike annealing methods, NS requires only a small number of
tuning parameters (just the number of particles $N$ is inherent to NS, although
most NS implementations have additional tuning parameters related to their
methods for sampling Equation~\ref{eq:constrained_prior}). Annealing methods
also tend to fail on phase change problems \citep{skilling} which can arise
in data analysis \citep{rjobject, exoplanet} as well as in statistical
mechanics, where they are more familiar.

In statistical mechanics, Equation~\ref{eqn:power_posterior} defines the
family of {\it canonical distributions}, usually written as:
\begin{eqnarray}
p(\xx; \beta) &=& \frac{\pi(\xx)\exp[-\beta E(\xx)]}{Z(\beta)}
\end{eqnarray}
where $E(\xx)$ is the energy function (analogous to minus the log likelihood
in the Bayesian case), and $\pi(\xx)$ is often uniform over
phase space (the set of possible positions and momenta of a collection of
particles) or configuration space (the set of possible positions of a collection
of particles). In this context we usually want to
compute $Z(\beta)$ as a function of $\beta$, which is called the
{\it partition function}. When the full phase space of positions and momenta
is included, the KL-divergence as a function of temperature,
$H(\beta)$, is the Gibbs entropy (up to a sign change and a constant factor).
A single run of NS can be used to reconstruct $Z(\beta)$ and $H(\beta)$,
allowing the study of
the properties of materials from first principles, based on hypotheses about
the atoms or molecules that make them up
\citep[e.g.][]{partay2010efficient, baldock2016determining}.
The NS exploration algorithm is invariant under
monotonic transformations of $L$ or $E$ and therefore there we can discuss the
algorithm in Bayesian or statistical mechanical terms without loss of
generality.

Throughout this paper we will refer to $L(\xx)$ or
$-E(\xx)$ as `objective functions',
by analogy with the terminology used when
discussing optimization methods. Although we are interested in sampling rather
than optimization, we still want to increase the values of $L$ or decrease
the values of $E$ relative to what is typical of the prior.

\section{Multiple objective functions}
In some inference and
statistical mechanics problems, there are two or more scalar functions of
$\xx$ that are relevant. Suppose our prior is $\pi(\xx)$ as before, and
we obtain information that fixes the expected values of two scalar
``likelihood'' functions of $\xx$, $L_1(\xx)$ and $L_2(\xx)$
(or equivalently, ``energy'' functions $E_1(\xx) = -\ln L_1(\xx)$ and
$E_2 = -\ln L_2(\xx)$). The
updated probability distribution that takes into account the constraints is
of the canonical form \citep{jaynes1957information}:
\begin{align}
p(\xx; \beta_1, \beta_2) &=
    \frac{\pi(\xx)L_1(\xx)^{\beta_1}L_2(\xx)^{\beta_2}}
         {Z(\beta_1, \beta_2)} \\
    &=
    \frac{\pi(\xx)\exp\left[-\beta_1E_1(\xx) - \beta_2E_2(\xx)\right]}
         {Z(\beta_1, \beta_2)}
\end{align}
where $\beta_1$ and $\beta_2$ are two `inverse temperatures' (the temperatures
themselves are $T_1 = \beta_1^{-1}$ and $T_2 = \beta_2^{-1}$).
In a Bayesian
context, this distribution (when $\beta_1 = \beta_2 = 1$) could be the
posterior distribution for parameters $\xx$
given two datasets, where $L_1$ and $L_2$ are the
likelihood functions for each dataset. In statistical mechanics, $E_1$ might
be the total energy and $E_2$ might be the total
angular momentum, or perhaps a term depending on an external field (so $\beta_2$
would then describe the strength of the external field).

If we are only interested in a single canonical distribution, for example
with $\beta_1 = 0.3$ and $\beta_2 = 0.7$, we can estimate its normalising
constant and by running standard Nested Sampling with objective function
$L(\xx) = L_1(\xx)^{\beta_1}L_2(\xx)^{\beta_2}$. However, usually we
are interested in a range of values for $\beta_1$ and $\beta_2$, and we
want to know the entire partition function $Z(\beta_1, \beta_2)$.
This work describes
progress towards solving this class of problems while maintaining the benefits
of Nested Sampling, such as the ability to cope with first-order phase
transitions. The algorithm is presented here for problems with two objective
functions, but the implementation allows for more than two (though the
performance is those cases is likely to be poor).


\

The prior $\pi(\xx)$ implies a certain prior for $L_1$ and $L_2$, which we
denote $\pi(L_1, L_2)$. An example of a prior is shown as the
probability density in Figure~\ref{fig:joint1}. The partition function
$Z(\beta_1, \beta_2)$ is a set of expected values with respect to this density.
Simple Monte Carlo sampling from $\pi$ will not work except for values of
$(\beta_1, \beta_2)$ where the canonical distribution remains similar to $\pi$.
We need a sampler that explores regions where $L_1$ and $L_2$ are high in order
to accurately estimate $Z(\beta_1, \beta_2)$.
%\begin{figure}[ht!]
%\centering
%\includegraphics[scale=0.5]{figures/joint1.pdf}
%\caption{\it An example of a prior for $L_1$ and $L_2$, which is implied by
%the prior $\pi(\xx)$. We have overplotted 25 points drawn from this prior.
%The estimated prior mass in the shaded rectangle is
%$\hat{X}(2, 1.5) = 5/25 = 0.2$.
%\label{fig:joint1}}
%\end{figure}

%Our proposed algorithm (introduced in Section~\ref{sec:algorithm}) makes use
%of the (two dimensional) cumulative distribution of the objective functions.
%For any values $(\ell_1, \ell_2)$ of the objective functions,
%the prior mass for which both $L_1 \geq \ell_1$ {\em and} $L_2 \geq \ell_2$ is:
%\begin{eqnarray}
%X(\ell_1, \ell_2) &=& \int \pi(L_1, L_2)
%\mathds{1}\left[L_1 \geq \ell_1, L_2 \geq \ell_2 \right]
% \, dL_1 \, dL_2.
%\end{eqnarray}
%Since $X$ is a probability, it must satisfy a product rule:
%\begin{eqnarray}
%X(\ell_1, \ell_2) = X(\ell_1)X(\ell_2 | \ell_1) = X(\ell_2)X(\ell_1 | \ell_2).
%\end{eqnarray}

%When the prior $\pi$ is approximated with a set of $N$
%particles $\{\xx_i\}_{i=1}^N$
%(Monte Carlo samples ``drawn from'' $\pi$),
%$X(\ell_1, \ell_2)$ can be approximated by:
%\begin{eqnarray}
%\hat{X}(\ell_1, \ell_2) &=&
%\frac{1}{N}
%\sum_{i=1}^N \mathds{1}\left[L_1(\xx_i) \geq \ell_1,
%L_2(\xx_i) \geq \ell_2\right].\label{eqn:corner_count}
%\end{eqnarray}
%Graphically, on a plot of $L_2$ vs. $L_1$, $\hat{X}(\ell_1, \ell_2)$
%is the fraction of particles in the rectangle whose upper right corner is at
%$\left(\ell_1, \ell_2\right)$. See Figure~\ref{fig:joint1} for an
%example.

\section{Unique Properties of Nested Sampling}
What is special about the NS sequence of distributions? 

Consider the Nested Sampling measure, proportional to the amount of time
NS spends in each region of the space. Since NS moves at a constant rate
in terms of $\log(X)$, this measure is proportional to
\begin{align}
m_{\rm NS} &\propto \frac{\pi(\xx)}{X(\xx)}.\label{eqn:measure}
\end{align}
The sequence of dead particles emitted by NS is then used to compute properties
of the posterior distribution, or other canonical distributions for temperatures
other than 1. Why does this work? Why does the NS measure of
Equation~\ref{eqn:ns_measure} allow the user to get a reasonable number of
samples for {\em any} canonical distribution?






We seek an algorithm for computing the partition function
$Z(\beta_1, \beta_2)$ for a range of values of the $\beta$s, from a single
run. To be a variant of Nested Sampling, the algorithm should satisfy
the following properties:
\begin{enumerate}
\item It should begin with $N$ points drawn from the prior $\pi(\xx)$.
\item It should seek to explore regions where the values of
$L_1(\xx)$ and $L_2(\xx)$ are higher than the prior $\pi(L_1, L_2)$
would typically imply.
\item The algorithm should consider a sequence of probability
distributions proportional to $\pi$, but restricted to smaller and smaller
domains for which the enclosed prior mass shrinks by a (roughly) constant and
known factor at each iteration.
\item The algorithm should be invariant to monotonic transformations of
$L_1$ and $L_2$, i.e. it should only depend on rankings of $L_1$ and $L_2$
values and not the values themselves.
\end{enumerate}
The standard Nested Sampling algorithm has these properties but for a
single objective function. The same is true of variants such as
Diffusive Nested Sampling \citep{dns, dnest4}.


Sequential Monte Carlo \citep{salomone2018unbiased}

%The introduction should briefly place the study in a broad context and highlight why it is important. It should define the purpose of the work and its significance. The current state of the research field should be reviewed carefully and key publications cited. Please highlight controversial and diverging hypotheses when necessary. Finally, briefly mention the main aim of the work and highlight the principal conclusions. As far as possible, please keep the introduction comprehensible to scientists outside your particular field of research. Citing a journal paper \cite{ref-journal}. And now citing a book reference \cite{ref-book}. Please use the command \citep{ref-journal} for the following MDPI journals, which use author-date citation: Administrative Sciences, Arts, Econometrics, Economies, Genealogy, Humanities, IJFS, JRFM, Languages, Laws, Religions, Risks, Social Sciences.
% 
%%%%%%%%%%%%%%%%%%%%%%%%%%%%%%%%%%%%%%%%%%
\section{Results}

This section may be divided by subheadings. It should provide a concise and precise description of the experimental results, their interpretation as well as the experimental conclusions that can be drawn.
\begin{quote}
This section may be divided by subheadings. It should provide a concise and precise description of the experimental results, their interpretation as well as the experimental conclusions that can be drawn.
\end{quote}

%%%%%%%%%%%%%%%%%%%%%%%%%%%%%%%%%%%%%%%%%%
\subsection{Subsection}

\subsubsection{Subsubsection}

Bulleted lists look like this:
\begin{itemize}[leftmargin=*,labelsep=5.8mm]
\item	First bullet
\item	Second bullet
\item	Third bullet
\end{itemize}

Numbered lists can be added as follows:
\begin{enumerate}[leftmargin=*,labelsep=4.9mm]
\item	First item 
\item	Second item
\item	Third item
\end{enumerate}

The text continues here.

\subsection{Figures, Tables and Schemes}

All figures and tables should be cited in the main text as Figure 1, Table 1, etc.

\begin{figure}[H]
\centering
\includegraphics[width=2 cm]{Definitions/logo-mdpi}
\caption{This is a figure, Schemes follow the same formatting. If there are multiple panels, they should be listed as: (\textbf{a}) Description of what is contained in the first panel. (\textbf{b}) Description of what is contained in the second panel. Figures should be placed in the main text near to the first time they are cited. A caption on a single line should be centered.}
\end{figure}   

\begin{table}[H]
\caption{This is a table caption. Tables should be placed in the main text near to the first time they are cited.}
\centering
%% \tablesize{} %% You can specify the fontsize here, e.g.  \tablesize{\footnotesize}. If commented out \small will be used.
\begin{tabular}{ccc}
\toprule
\textbf{Title 1}	& \textbf{Title 2}	& \textbf{Title 3}\\
\midrule
entry 1		& data			& data\\
entry 2		& data			& data\\
\bottomrule
\end{tabular}
\end{table}

\subsection{Formatting of Mathematical Components}

This is an example of an equation:

\begin{equation}
a + b = c
\end{equation}
%% If the documentclass option "submit" is chosen, please insert a blank line before and after any math environment (equation and eqnarray environments). This ensures correct linenumbering. The blank line should be removed when the documentclass option is changed to "accept" because the text following an equation should not be a new paragraph. 

Please punctuate equations as regular text. Theorem-type environments (including propositions, lemmas, corollaries etc.) can be formatted as follows:
%% Example of a theorem:
\begin{Theorem}
Example text of a theorem.
\end{Theorem}

The text continues here. Proofs must be formatted as follows:

%% Example of a proof:
\begin{proof}[Proof of Theorem 1]
Text of the proof. Note that the phrase `of Theorem 1' is optional if it is clear which theorem is being referred to.
\end{proof}
The text continues here.

%%%%%%%%%%%%%%%%%%%%%%%%%%%%%%%%%%%%%%%%%%
\section{Discussion}

Authors should discuss the results and how they can be interpreted in perspective of previous studies and of the working hypotheses. The findings and their implications should be discussed in the broadest context possible. Future research directions may also be highlighted.

%%%%%%%%%%%%%%%%%%%%%%%%%%%%%%%%%%%%%%%%%%
\section{Materials and Methods}

Materials and Methods should be described with sufficient details to allow others to replicate and build on published results. Please note that publication of your manuscript implicates that you must make all materials, data, computer code, and protocols associated with the publication available to readers. Please disclose at the submission stage any restrictions on the availability of materials or information. New methods and protocols should be described in detail while well-established methods can be briefly described and appropriately cited.

Research manuscripts reporting large datasets that are deposited in a publicly available database should specify where the data have been deposited and provide the relevant accession numbers. If the accession numbers have not yet been obtained at the time of submission, please state that they will be provided during review. They must be provided prior to publication.

Interventionary studies involving animals or humans, and other studies require ethical approval must list the authority that provided approval and the corresponding ethical approval code. 

%%%%%%%%%%%%%%%%%%%%%%%%%%%%%%%%%%%%%%%%%%
\section{Conclusions}

This section is not mandatory, but can be added to the manuscript if the discussion is unusually long or complex.

%%%%%%%%%%%%%%%%%%%%%%%%%%%%%%%%%%%%%%%%%%
\section{Patents}
This section is not mandatory, but may be added if there are patents resulting from the work reported in this manuscript.

%%%%%%%%%%%%%%%%%%%%%%%%%%%%%%%%%%%%%%%%%%
\vspace{6pt} 

%%%%%%%%%%%%%%%%%%%%%%%%%%%%%%%%%%%%%%%%%%
%% optional
%\supplementary{The following are available online at \linksupplementary{s1}, Figure S1: title, Table S1: title, Video S1: title.}

% Only for the journal Methods and Protocols:
% If you wish to submit a video article, please do so with any other supplementary material.
% \supplementary{The following are available at \linksupplementary, Figure S1: title, Table S1: title, Video S1: title. A supporting video article is available at doi: link.}

%%%%%%%%%%%%%%%%%%%%%%%%%%%%%%%%%%%%%%%%%%
\authorcontributions{For research articles with several authors, a short paragraph specifying their individual contributions must be provided. The following statements should be used ``Conceptualization, X.X. and Y.Y.; Methodology, X.X.; Software, X.X.; Validation, X.X., Y.Y. and Z.Z.; Formal Analysis, X.X.; Investigation, X.X.; Resources, X.X.; Data Curation, X.X.; Writing—Original Draft Preparation, X.X.; Writing—Review \& Editing, X.X.; Visualization, X.X.; Supervision, X.X.; Project Administration, X.X.; Funding Acquisition, Y.Y.'', please turn to the \href{http://img.mdpi.org/data/contributor-role-instruction.pdf}{CRediT taxonomy} for the term explanation. Authorship must be limited to those who have contributed substantially to the work reported. }

%%%%%%%%%%%%%%%%%%%%%%%%%%%%%%%%%%%%%%%%%%
\funding{Please add: ``This research received no external funding'' or ``This research was funded by [name of funder] grant number [xxx].'' Check carefully that the details given are accurate and use the standard spelling of funding agency names at \url{https://search.crossref.org/funding}, any errors may affect your future funding.}

%%%%%%%%%%%%%%%%%%%%%%%%%%%%%%%%%%%%%%%%%%
\acknowledgments{In this section you can acknowledge any support given which is not covered by the author contribution or funding sections. This may include administrative and technical support, or donations in kind (e.g. materials used for experiments).}

%%%%%%%%%%%%%%%%%%%%%%%%%%%%%%%%%%%%%%%%%%
\conflictsofinterest{Declare conflicts of interest or state ``The authors declare no conflict of interest.'' Authors must identify and declare any personal circumstances or interest that may be perceived as inappropriately influencing the representation or interpretation of reported research results. Any role of the funding sponsors in the design of the study; in the collection, analyses or interpretation of data; in the writing of the manuscript, or in the decision to publish the results must be declared in this section. If there is no role, please state ``The founding sponsors had no role in the design of the study; in the collection, analyses, or interpretation of data; in the writing of the manuscript, and in the decision to publish the results''.} 

%%%%%%%%%%%%%%%%%%%%%%%%%%%%%%%%%%%%%%%%%%
%% optional
\abbreviations{The following abbreviations are used in this manuscript:\\

\noindent 
\begin{tabular}{@{}ll}
MDPI & Multidisciplinary Digital Publishing Institute\\
DOAJ & Directory of open access journals\\
TLA & Three letter acronym\\
LD & linear dichroism
\end{tabular}}

%%%%%%%%%%%%%%%%%%%%%%%%%%%%%%%%%%%%%%%%%%
%% optional
\appendixtitles{no} %Leave argument "no" if all appendix headings stay EMPTY (then no dot is printed after "Appendix A"). If the appendix sections contain a heading then change the argument to "yes".
\appendixsections{multiple} %Leave argument "multiple" if there are multiple sections. Then a counter is printed ("Appendix A"). If there is only one appendix section then change the argument to "one" and no counter is printed ("Appendix").
\appendix
\section{}
\subsection{}
The appendix is an optional section that can contain details and data supplemental to the main text. For example, explanations of experimental details that would disrupt the flow of the main text, but nonetheless remain crucial to understanding and reproducing the research shown; figures of replicates for experiments of which representative data is shown in the main text can be added here if brief, or as Supplementary data. Mathematical proofs of results not central to the paper can be added as an appendix.

\section{}
All appendix sections must be cited in the main text. In the appendixes, Figures, Tables, etc. should be labeled starting with `A', e.g., Figure A1, Figure A2, etc. 

%%%%%%%%%%%%%%%%%%%%%%%%%%%%%%%%%%%%%%%%%%
% Citations and References in Supplementary files are permitted provided that they also appear in the reference list here. 

%=====================================
% References, variant A: internal bibliography
%=====================================
\reftitle{References}

%\bibliographystyle{mdpi}
\bibliography{references.bib/references}

% The following MDPI journals use author-date citation: Arts, Econometrics, Economies, Genealogy, Humanities, IJFS, JRFM, Laws, Religions, Risks, Social Sciences. For those journals, please follow the formatting guidelines on http://www.mdpi.com/authors/references
% To cite two works by the same author: \citeauthor{ref-journal-1a} (\citeyear{ref-journal-1a}, \citeyear{ref-journal-1b}). This produces: Whittaker (1967, 1975)
% To cite two works by the same author with specific pages: \citeauthor{ref-journal-3a} (\citeyear{ref-journal-3a}, p. 328; \citeyear{ref-journal-3b}, p.475). This produces: Wong (1999, p. 328; 2000, p. 475)

%=====================================
% References, variant B: external bibliography
%=====================================
%\externalbibliography{yes}
%\bibliography{your_external_BibTeX_file}

%%%%%%%%%%%%%%%%%%%%%%%%%%%%%%%%%%%%%%%%%%
%% optional
\sampleavailability{Samples of the compounds ...... are available from the authors.}

%% for journal Sci
%\reviewreports{\\
%Reviewer 1 comments and authors’ response\\
%Reviewer 2 comments and authors’ response\\
%Reviewer 3 comments and authors’ response
%}

%%%%%%%%%%%%%%%%%%%%%%%%%%%%%%%%%%%%%%%%%%
\end{document}

