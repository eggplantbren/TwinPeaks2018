\documentclass[a4paper, 12pt]{article}
\usepackage{amsmath}
\usepackage{amssymb}
\usepackage{dsfont}
\usepackage[left=2cm, right=2cm, bottom=3cm, top=2cm]{geometry}
\usepackage{graphicx}
\usepackage[utf8]{inputenc}
\usepackage{microtype}
\usepackage{natbib}

\newcommand{\bell}{\boldsymbol{\ell}}
\newcommand{\btheta}{\boldsymbol{\theta}}
\newcommand{\given}{\,|\,}

\title{TwinPeaks2018}
\author{Brendon J. Brewer}
%\date{}

\begin{document}
\maketitle

%\abstract{\noindent Abstract}

% Need this after the abstract
\setlength{\parindent}{0pt}
\setlength{\parskip}{8pt}

The prior is $\pi(\btheta)$ and the scalars are
$f(\btheta)$ and $g(\btheta)$. Letting $x=f(\btheta)$ and $y=g(\btheta)$,
the implied prior for $x$ and $y$ will be written $\pi(x, y)$.

Consider a family of thresholded distributions, indexed by threshold
values $(x^*, y^*)$, given by
\begin{align}
p(x, y; x^*, y^*) &= \frac{\pi(x, y)\mathds{1}(x > x^*, y > y^*)}
                          {X(x^*, y^*)}
\end{align}
The normalising constant is the probability in the `upper right corner':
\begin{align}
X(x^*, y^*) &= \int_{y^*}^{\infty} \int_{x^*}^{\infty} \pi(x, y) \, dx \, dy.
\end{align}

The Kullback-Leibler divergence from the prior to a thresholded distribution
is
\begin{align}
D_{\rm KL}\left(p(x, y; x^*, y^*) \,||\, \pi(x, y)\right) &=
    \log X(x^*, y^*).
\end{align}

\section{Nested Sampling distribution}
Consider the one-scalar situation (i.e., $f(x) \equiv$ log likelihood),
which is the case in standard NS. The Nested Sampling distribution is
\begin{align}
p_{\rm NS}(x) &= \frac{\pi(x)}{C X(x)}.
\end{align}
This is the distribution that `stretches out' the prior (which is a
reflected exponential for $\ln X$) into a uniform distribution for $\ln X$.
\begin{align}
p_{\rm NS}(x) &= \frac{\pi(x)}{C X(x)}.
\end{align}

The KL divergence from $p_{\rm NS}$ to a constrained distribution
$p(x; x^*) = \pi(x)\mathds{1}(x > x^*)/X(x^*)$ is
\begin{align}
D_{\rm KL}\left(p(x; x^*) \,||\, p_{\rm NS}\right)
    &= \int p(x; x^*)
            \log \left[\frac{p_{\rm NS}}{p_{\rm NS}}\right] \, dx \\
    &= \int \pi(x)\mathds{1}(x > x^*)/X(x^*)
            \log \left[\frac{\pi(x)\mathds{1}(x > x^*)/X(x^*)}
                            {\frac{\pi(x)}{C X(x)}}\right] \, dx \\
    &= \int \pi(x)\mathds{1}(x > x^*)/X(x^*)
            \log \left[\frac{\pi(x)\mathds{1}(x > x^*)/X(x^*)}
                            {\frac{\pi(x)}{C X(x)}}\right] \, dx \\
    &= \log C + \int_{x^*}^{\infty} \frac{\pi(x)}{X(x^*)}
            \log \left[\frac{X(x)}{X(x^*)}\right] \, dx.
\end{align}

Let's see how this varies as you change the thresholded distribution
by varying $x^*$:
\begin{align}
\frac{d}{dx^*} D_{\rm KL}\left(p(x; x^*) \,||\, p_{\rm NS}\right)
    &= \frac{d}{dx^*} \log C +
        \frac{d}{dx^*}\int_{x^*}^{\infty} \frac{\pi(x)}{X(x^*)}
            \log \left[\frac{X(x)}{X(x^*)}\right] \, dx \\
    &= \frac{d}{dx^*}\int_{x^*}^{\infty} \frac{\pi(x)}{X(x^*)}
            \log \left[\frac{X(x)}{X(x^*)}\right] \, dx \\
    &= \frac{\pi(x^*)}{X(x^*)}
            \log \left[\frac{X(x^*)}{X(x^*)}\right] \, dx \\
    &= 0.
\end{align}
That is, it doesn't change! The KL divergence
{\em from the distribution actually sampled} to
{\em any thresholded distribution} is the same!




\section{Nested Sampling distribution --- TwinPeaks case}

Consider the following distribution:
\begin{align}
p_{\rm NS}(x, y) &= \frac{\pi(x, y)}{C X(x, y)}.
\end{align}
Compared to the prior $\pi$, this puts more weight at high values of
$x$ and $y$.

The KL divergence from $p_{\rm NS}$ to $p(x, y ; x^*, y^*)$ is
\begin{align}
D_{\rm KL}\left(p(x, y; x^*, y^*) \,||\, p_{\rm NS}(x, y)\right)
&=
\int_{y^*}^{\infty} \int_{x^*}^{\infty}
    \frac{\pi(x, y)}{X(x^*, y^*)}
    \log \left[\frac{C X(x, y)}{X(x^*, y^*)}\right]
    \, dx \, dy \\
&=
\frac{1}{X(x^*, y^*)}
\int_{y^*}^{\infty} \int_{x^*}^{\infty}
    \pi(x, y)
    \log \left[\frac{C X(x, y)}{X(x^*, y^*)}\right]
    \, dx \, dy \\
&=
\frac{\int_{y^*}^{\infty} \int_{x^*}^{\infty}
     \pi(x, y)
     \log \left[C X(x, y)\right]
     \, dx \, dy - X(x^*, y^*)\log X(x^*, y^*)}
     {X(x^*, y^*)}
\end{align}



\end{document}

