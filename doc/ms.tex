\documentclass[a4paper, 12pt]{article}
\usepackage{amsmath}
\usepackage{amssymb}
\usepackage{dsfont}
\usepackage[left=2cm, right=2cm, bottom=3cm, top=2cm]{geometry}
\usepackage{graphicx}
\usepackage[utf8]{inputenc}
\usepackage{microtype}
\usepackage{natbib}

\newcommand{\bell}{\boldsymbol{\ell}}
\newcommand{\btheta}{\boldsymbol{\theta}}
\newcommand{\given}{\,|\,}

\title{TwinPeaks2018}
\author{Brendon J. Brewer}
%\date{}

\begin{document}
\maketitle

%\abstract{\noindent Abstract}

% Need this after the abstract
\setlength{\parindent}{0pt}
\setlength{\parskip}{8pt}

The prior is $\pi(\btheta)$ and the scalars are
$\left\{ f_i(\btheta) \right\}$
for $i \in \{1, 2, ..., M\}$. In most examples $M$ will be 2.
Consider a family of thresholded distributions, indexed by threshold
values $\bell=\left\{\ell_i\right\}$, proportional to
\begin{align}
p(\btheta; \bell) &\propto
    \pi(\btheta)
    \prod_{i=1}^M \mathds{1}\left(f_i(\btheta) > \ell_i\right).
\end{align}
The normalising constant is the probability in the `upper right corner'
(in 2D):
\begin{align}
X(\bell) &= \int_{f_i(\btheta) > \ell_i, \forall_i} \pi(\btheta) \, d\btheta
\end{align}


\end{document}

